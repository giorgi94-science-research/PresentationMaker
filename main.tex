\documentclass[12pt]{article}
\usepackage{fontspec}

% \setmainfont[
%     Extension = .ttf,
%     Path = fonts/,
%     BoldFont = Myfont-bold,
%     BoldItalicFont = Myfont-bolditalic,
%     ItalicFont = Myfont-italic
% ]{FiraGO}

\setmainfont{FiraGO}
\usepackage[english, georgian]{babel}
\usepackage{mypresentation}





\theoremstyle{colored}
\newtheorem{theorem}{თეორემა}



% \newenvironment{slide}[1]{
%     \ifx&#1&
%     \newpage
%     \else
%     \newpage\section{#1}
%     \fi
% }{}

\setcounter{page}{0}

\newenvironment{slide}[1]{\newpage #1}{\vfill\rightline{\thepage}}



\begin{document}


\title{\Large\color{primary} Computation of Generalized Modulus of an $n$-gon}
\author{\normalsize გიორგი კაკულაშვილი}
\date{\normalsize \today}
\maketitle


% \begin{slide}
%     \tableofcontents
% \end{slide}


\begin{slide}
    \slidetitle{Sample frame tilte}

    This is intersting

    \[
        e^x = \sum_{n=0}^{\infty} \frac{x^n}{n!} = 1 + x + \frac{x^2}{2} + \cdots
    \]

    \begin{theorem}
        This is the great theorem $c^2=a^2+b^2$
    \end{theorem}

\end{slide}


\begin{slide}
    \slidetitle{This is other}
    This second slide 

    
\end{slide}

\begin{slide}
    This third slide
\end{slide}


\begin{slide}
    This third slide
\end{slide}



\end{document}